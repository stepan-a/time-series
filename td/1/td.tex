\documentclass[10pt,a4paper,notitlepage]{article}
\usepackage{amsmath}
\usepackage{amssymb}
\usepackage{amsbsy}
\usepackage{float}
\usepackage[french]{babel}
\usepackage{graphicx}
\usepackage{nicefrac}

\usepackage[utf8x]{inputenc}
\usepackage[T1]{fontenc}
\usepackage{palatino}

 \usepackage[active]{srcltx}
\usepackage{scrtime}

\newcommand{\exercice}[1]{\textsc{\textbf{Exercice}} #1}
\newcommand{\question}[1]{\textbf{(#1)}}
\setlength{\parindent}{0cm}

\begin{document}

\title{\textsc{Séries temporelles\\ \small{(Fiche de TD n°1)}}}
\author{Stéphane Adjemian\thanks{Université du Maine, Gains. \texttt{stephane DOT adjemian AT univ DASH lemans DOT fr}}}
\date{Le \today\ à \thistime}

\maketitle

\exercice{1} Soient $X$ et $Y$ deux variables aléatoires. Montrer que la covariance entre ces deux variables peut s'écrire sous la forme :
\[
\mathbb Cov(X,Y) = \mathbb E [XY] - \mathbb E [X]\mathbb E [Y]
\]

\bigskip

\exercice{2} Soit le processus MA(1) :
\[
y_t = 1 + \epsilon_t - \frac{1}{2}\epsilon_{t-1}
\]
Calculer la fonction d'autocorrélation et donner une représentation graphique.

\bigskip

\exercice{3} Montrer que l'autocorrélation d'ordre 1 d'un processus MA(1), $y_t=\epsilon_t+\theta\epsilon_{t-1}$, ne peut être supérieure à $\nicefrac{1}{2}$ (en valeur absolue). Pourquoi est-il raisonnable de travailler sous l'hypothèse que le paramètre $\theta$ est compris entre -1 et 1 ?

\bigskip

\exercice{4} Soit le processus MA(2) :
\[
y_t = \epsilon_t - \theta_1\epsilon_{t-1} - \theta_2\epsilon_{t-2}
\]
Calculer la fonction d'autocovariance et la fonction d'autocorrélation.

\bigskip

\exercice{5} La densité spectrale d'un processus stochastique $\{y_t\}$ dont la fonction d'autocovariance est notée $\gamma_y (h)$ est définie par :
\[
f_y(\omega) = (2\pi)^{-1}\sum_{h=-\infty}^{\infty}\gamma_y(h)e^{-i \omega h}
\]
On suppose que le processus stochastique est tel que la somme soit définie pour tout $\omega$ réel. \textbf{(1)} Montrer que la fonction $f_y$ est réelle, symétrique sur l'intervalle $[-\pi,\pi]$ et périodique. \textbf{(2)} Représenter graphiquement la densité spectrale d'un bruit blanc. \textbf{(3)} Représenter la densité spectrale d'un MA(1). \textbf{(4)} Même question avec un MA(2).


\end{document}


% Exo 2.

 
