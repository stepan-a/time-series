% stephane [DOT] adjemian [AT] univ [DASH] lemans [DOT] fr
\documentclass[10pt,a4paper,notitlepage,twocolumn]{article}
\synctex=1
\usepackage{amsmath}
\usepackage{amssymb}
\usepackage{amsbsy}
\usepackage[T1]{fontenc}
\usepackage[utf8x]{inputenc}
\usepackage{palatino}
\usepackage{scrtime}
\usepackage[frenchb]{babel}
\usepackage{float}

%%%%%%%%%%%%%%%%%%%%%%%%%%%%%%%%%%%%%%%%%%%%%%%%%%%%%%%%%%%%%%%%%%%%%%%%%%%%%%%%%%%%%%%%%%%%%%%%%%%%

\newcommand{\exercice}[1]{\textsc{\textbf{Exercice}} #1}
\newcommand{\question}[1]{\textbf{(#1)}}
\setlength{\parindent}{0cm}



\begin{document}


\title{\textsc{Séries temporelles}}
\author{\textsc{Université du Maine (Partiel, L3)}}
\date{}


\maketitle


\exercice{1} Soit $\{\varepsilon_t\}$ un bruit blanc d'espérance nulle
et de variance $\sigma_{\varepsilon}^2>0$, déterminer quels sont les
processus stochastiques pour lesquels $y_t$ et $y_{t-2}$ sont
indépendants:
\begin{enumerate}
\item $y_t = \frac{1}{2}y_{t-1} + \varepsilon_t$
\item $y_t = 1 + \varepsilon_t - \frac{1}{10} \varepsilon_{t-1}$
\item $y_t = 3 + \varepsilon_t + \frac{2}{10} \varepsilon_{t-1} - \frac{1}{10} \varepsilon_{t-2}$
\end{enumerate}
Justifier vos réponse.\newline

\bigskip

\exercice{2} Donner l'expression des vraisemblances exacte et
conditionnelle d'un processus AR(1) d'espérance nulle. On notera
$\mathcal Y_T \equiv \{y_1,y_2,\dots,y_T\}$ l'échantillon et on
supposera que les innovations sont normalement distribuées d'espérance
nulle et de variance $\sigma_{\epsilon}^2$. Quel est le rapport avec
l'estimateur des Moindres Carrés Ordinaires ?\newline

\bigskip

\exercice{3} Supposons que $\{y_t,t\in\mathbb Z\}$ soit un ARMA($2,1$) de la forme :

\[
y_t = 1 + \frac{1}{3}y_{t-1} - \frac{1}{3}y_{t-2} + \varepsilon_t - \frac{1}{3} \varepsilon_{t-1}
\]
avec $\varepsilon_t$ un bruit blanc d'espérance nulle et de variance
1. \question{1} Le processus est-il asymptotiquement stationnaire au
second ordre et inversible~?  Justifier votre réponse.\newline

On suppose maintenant que le processus est stationnaire au second
ordre. \question{2} Quelles sont les implications de cette hypothèse
sur les moments d'ordre 1 et 2 ? \question{3} Calculer l'espérance (on
notera $\mu$). \question{4} Calculer les autocovariances d'ordre 0, 1
et 2 (on notera $\gamma(0)$, $\gamma(1)$ et $\gamma(2)$). \question{5}
Calculer l'autocovariance d'ordre 3 (on notera
$\gamma(3)$). \question{6} Calculer l'autocovariance d'ordre h (on
notera $\gamma(h)$) pour tout $h>2$.  \question{7} Définir la fonction
d'autocorrélation.\newline

\bigskip

\exercice{4} Calculer les moments d'ordre 1 et 2 d'un processus
ARMA(1,2) que nous supposerons stationnaire et inversible :
\[
y_t = c + \varphi_1 y_{t-1} + \varepsilon_t - \theta_1 \varepsilon_{t-1} - \theta_2 \varepsilon_{t-2}
\]
où $\varepsilon_t$ est un bruit blanc d'espérance nulle et de variance
$\sigma^2$.\newline

\bigskip

\exercice{5} Un modèle macroéconomique nous dit que la production est
déterminée par une fonction de production Cobb-Douglas ; le logarithme
de la production est une fonction linéaire du stock de
capital en logarithme:
\[
y_t = \alpha k_{t} + a_t
\]
où $a_t$ est le logarithme de la productivité, dont la dynamique est
donnée par un modèle AR(1):
\[
a_t = \rho a_{t-1} + \varepsilon_t
\]
avec $\varepsilon_t$ une variable aléatoire d'espérance nulle et de
variance $\sigma_{\varepsilon}^2$. Le paramètre $\rho$ est tel que la
productivité (en logarithme) est stationnaire, le paramètre
$\alpha\in]0,1[$ est l'élasticité de la production par rapport au
stock de capital physique. La dynamique de ce dernier résulte des
choix d'épargne des ménages. On admet qu'elle est donnée par:
\[
k_t = \eta_{kk} k_{t-1} + \eta_{ka}a_{t-1}
\]
\question{1} Montrer que $y_t$ est un processus
ARMA(2,1). \emph{\textbf{Indice :} Réécrire le modèle à l'aide de polynômes retard.} \question{2} Calculer les racines de la partie
autorégressive de ce processus ARMA(2,1) et déduire une condition sur
$\eta_{kk}$ assurant la stationarité de la production.

\end{document}

%%% Local Variables:
%%% mode: latex
%%% TeX-master: t
%%% End:
