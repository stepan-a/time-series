% stephane [DOT] adjemian [AT] univ [DASH] lemans [DOT] fr
\documentclass[10pt,a4paper,notitlepage,twocolumn]{article}
\usepackage{amsmath}
\usepackage{amssymb}
\usepackage{amsbsy}
\usepackage[T1]{fontenc}
\usepackage[utf8x]{inputenc}
\usepackage{palatino}
\usepackage{scrtime}
\usepackage[frenchb]{babel}
\usepackage{float}


\newcommand{\exercice}[1]{\textsc{\textbf{Exercice}} #1}
\newcommand{\question}[1]{\textbf{(#1)}}
\setlength{\parindent}{0cm}

\begin{document}

\title{\textsc{Séries temporelles}}
\author{Stéphane Adjemian}
\date{Mardi 11 juin 2024}

\maketitle

\exercice{1} Soient les fonctions d'auto-corrélation et de corrélation
partielle suivantes~:
\begin{table}[H]
  \centering
  \begin{tabular}{r|cccccc}
    \hline\hline
    $h$ & 0 & 1 & 2 & 3 & 4 & 5 \\\hline
    $\rho (h)$ & 1 & 0,70 & 0,00 & 0,00 & 0,00 & 0,00\\
    $r (h)$ & -- & 0,70 & 0,49 & 0,34 & 0,24 & 0,17\\
    \hline\hline
  \end{tabular}
\end{table}
\noindent Ces  fonctions sont générées par  un processus ARMA($p$,$q$)
avec $p\geq 0$, $q\geq 0$ et $p+q \leq 2$. En
justifiant  votre réponse,  déterminez  la forme  du  processus qui  a
généré $\rho(h)$ et $r(h)$. Que pouvez-vous dire des paramètres du ce
modèle~?

\bigskip
\bigskip


\exercice{2} Soit le processus stochastique défini par~:
\[
  y_t = \varepsilon_t - \theta \varepsilon_{t-2}
\]
avec $\varepsilon_t$ un bruit blanc d'espérance nulle et de
variance $\sigma_{\varepsilon}^2$. \textbf{(1)} Quel est le nom de ce
processus~? \textbf{(2)} Est-il possible d'estimer les paramètres de
ce modèle ($\theta$ et $\sigma_{\varepsilon}^2$) par les MCO~?
\textbf{(3)} On note $\mathcal Y_T = \{y_1,\dots,y_T\}$ l'échantillon,
donner l'expression de la vraisemblance exacte. \textbf{(4)} Comment
évaluer la vraisemblance conditionnelle~?

\bigskip
\bigskip

\exercice{3} Supposons que $\{y_t,t\in\mathbb Z\}$ soit un ARMA($1,1$) de la forme~:

\[
y_t = 1 + \frac{2}{3}y_{t-1} + \varepsilon_t - \frac{1}{3} \varepsilon_{t-1}
\]
avec $\varepsilon_t$ un bruit blanc d'espérance nulle et de variance 1.\newline

\question{1} Le processus est-il asymptotiquement stationnaire au second ordre
et inversible~? Justifier votre réponse. \question{2} On suppose dans la suite
que le processus est stationnaire au second ordre. Quelles sont les implications
de cette hypothèse sur les moments d'ordre 1 et 2~? \question{3} Calculer
l'espérance (on notera $\mu$). \question{4} Calculer les autocovariances d'ordre
0 et 1 (on notera $\gamma(0)$ et $\gamma(1)$). \question{5} Calculer
l'autocovariance d'ordre 2 (on notera $\gamma(2)$). \question{6} Calculer
l'autocovariance d'ordre $h$ (on notera $\gamma(h)$) pour tout $h>2$.
\question{7} Définir la fonction d'autocorrélation.

\bigskip
\bigskip

\exercice{4} Supposons que $\{y_t,t\in\mathbb Z\}$ soit un ARMA($1,2$) stationnaire et inversible de la forme~:

\[
y_t = c + \varphi y_{t-1} + \varepsilon_t - \theta_1 \varepsilon_{t-1} - \theta_2 \varepsilon_{t-2}
\]
avec $\varepsilon_t$ un bruit blanc d'espérance nulle et de
variance $\sigma_{\varepsilon}^2$. Calculer les moments d'ordre 1 et 2
de ce processus stochastique.\newline

\end{document}
