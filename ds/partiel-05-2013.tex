% stephane [DOT] adjemian [AT] univ [DASH] lemans [DOT] fr
\documentclass[10pt,a4paper,notitlepage,onecolumn]{article}
\usepackage{amsmath}
\usepackage{amssymb}
\usepackage{amsbsy}
\usepackage[T1]{fontenc}
\usepackage[utf8x]{inputenc}
\usepackage{palatino}
\usepackage{scrtime}
\usepackage[frenchb]{babel}
\usepackage{float}

%%%%%%%%%%%%%%%%%%%%%%%%%%%%%%%%%%%%%%%%%%%%%%%%%%%%%%%%%%%%%%%%%%%%%%%%%%%%%%%%%%%%%%%%%%%%%%%%%%%%

\newcommand{\exercice}[1]{\textsc{\textbf{Exercice}} #1}
\newcommand{\question}[1]{\textbf{(#1)}}
\setlength{\parindent}{0cm}



\begin{document}


\title{\textsc{Séries temporelles}}
\author{\textsc{Université du Maine (Partiel, L3)}}
\date{}


\maketitle

\exercice{1} Soient les fonctions d'auto-corrélation et de corrélation
partielle suivantes
\begin{table}[H]
  \centering
  \begin{tabular}{r|cccccc}
    \hline\hline
    $h$ & 0 & 1 & 2 & 3 & 4 & 5 \\\hline
    $\rho (h)$ & 1 & 0,70 & 0,49 & 0,34 & 0,24 & 0,17\\
    $r (h)$ & -- & 0,70 & 0,00 & 0,00 & 0,00 & 0,00\\
    \hline\hline
  \end{tabular}
\end{table}
\noindent Ces  fonctions sont générées par  un processus ARMA($p$,$q$)
avec $p$ égal à zéro ou un, $q$ égal  à zéro ou un et $p+q \leq 2$. En
justifiant  votre réponse,  déterminez  la forme  du  processus qui  a
généré $\rho(h)$ et $r(h)$. Que  pouvez-vous dire des paramètres du ce
modèle~?

\bigskip
\bigskip


\exercice{2}  Montrez,  pour  un  modèle  AR(1),  l'équivalence  entre
l'estimateur  des  MCO et  l'estimateur  du  maximum de  vraisemblance
conditionnelle. Donnez l'expression de l'estimateur des MCO de $\rho$,
le paramètre autorégressif.

\bigskip
\bigskip

\exercice{3} Supposons que $\{y_t,t\in\mathbb Z\}$ soit un ARMA($1,1$) de la forme :

\[
y_t = \frac{1}{2}y_{t-1} + \varepsilon_t - \frac{1}{3} \varepsilon_{t-1}
\]
avec $\varepsilon_t$ un bruit blanc d'espérance nulle et de variance 1.\newline

\question{1}   Le    processus   est-il   asymptotiquement
stationnaire  au   second  ordre   et  inversible~?   Justifiez  votre
réponse.\newline

On suppose  que les conditions initiales sont  telles que le
processus est stationnaire au second ordre.\newline

\question{2} Quelles sont les implications de cette hypothèse sur les moments d'ordre 1 et 2 ? \question{3} Calculez l'espérance (on notera $\mu$ l'espérance). \question{4} Calculez les autocovariances d'ordre 0 et 1 (on notera $\gamma(0)$ et $\gamma(1)$). \question{5} Calculez l'autocovariance d'ordre 2 (on notera $\gamma(2)$). \question{6} Calculez l'autocovariance d'ordre h (on notera $\gamma(h)$) pour tout $h>2$.

\bigskip
\bigskip

\exercice{4} Calculez les moments d'ordre 1 et 2 d'un processus ARMA(2,2) que nous supposerons stationnaire :
\[
y_t = c + \varphi_1 y_{t-1} + \varphi_2 y_{t-2} + \varepsilon_t - \theta_1 \varepsilon_{t-1} - \theta_2 \varepsilon_{t-2}
\]
où $\varepsilon_t$ est un bruit blanc d'espérance nulle et de variance $\sigma^2$. 

\end{document}
