% stephane [DOT] adjemian [AT] univ [DASH] lemans [DOT] fr
\documentclass[10pt,a4paper,notitlepage,twocolumn]{article}
\usepackage{amsmath}
\usepackage{amssymb}
\usepackage{amsbsy}
\usepackage[T1]{fontenc}
\usepackage[utf8x]{inputenc}
\usepackage{palatino}
\usepackage{scrtime}
\usepackage[frenchb]{babel}
\usepackage{float}
\usepackage{nicefrac}

%%%%%%%%%%%%%%%%%%%%%%%%%%%%%%%%%%%%%%%%%%%%%%%%%%%%%%%%%%%%%%%%%%%%%%%%%%%%%%%%%%%%%%%%%%%%%%%%%%%%

\newcommand{\exercice}[1]{\textsc{\textbf{Exercice}} #1}
\newcommand{\question}[1]{\textbf{(#1)}}
\setlength{\parindent}{0cm}



\begin{document}


\title{\textsc{Séries temporelles}}
\author{\textsc{Université du Mans (Examen, L3)}}
\date{}

\pagenumbering{gobble}

\maketitle

\exercice{1} Soient les fonctions d'autocorrélation et de d'autocorrélation
partielle suivantes :
\begin{table}[H]
  \centering
  \begin{tabular}{r|cccccc}
    \hline\hline
    $h$ & 0 & 1 & 2 & 3 & 4 & 5 \\\hline
    $\rho (h)$ & 1 & .9 & .81 & .729 & .6561 & .59049\\
    $r (h)$ & -- & .9 & 0,00 & 0,00 & 0,00 & 0,00\\
    \hline\hline
  \end{tabular}
\end{table}
\noindent Ces fonctions sont générées par un processus ARMA($p$,$q$) avec
$p\geq 0$, $q\geq 0$ et $p+q \leq 2$. \question{1} En justifiant votre réponse,
déterminez la forme du processus qui a généré $\rho(h)$ et $r(h)$. \question{2}
Que pouvez-vous dire des paramètres de ce modèle~? \question{3} En supposant que
la variance de l'innovation du modèle est égale à $\nicefrac{1}{10}$, calculez
la variance du processus stochastique.\newline

\bigskip

\exercice{2} Soit $\{\varepsilon_t\}$ un bruit blanc d'espérance nulle
et de variance $\sigma_{\varepsilon}^2>0$, déterminer quels sont les
processus stochastiques pour lesquels $y_t$ et $y_{t-2}$ sont
indépendants:
\begin{enumerate}
\item $y_t = \frac{1}{2}y_{t-1} + \varepsilon_t$
\item $y_t = 1 + \varepsilon_t - \frac{1}{4} \varepsilon_{t-1}$
\item $y_t = 3 + \varepsilon_t + \frac{2}{10} \varepsilon_{t-1} - \frac{1}{10} \varepsilon_{t-2}$
\end{enumerate}
Justifier vos réponse.\newline

\bigskip

\exercice{3} Supposons que $\{y_t,t\in\mathbb Z\}$ soit un ARMA($1,1$) de la forme :
\[
y_t = 1 + \frac{2}{3}y_{t-1} + \varepsilon_t - \frac{1}{3} \varepsilon_{t-1}
\]
avec $\varepsilon_t$ un bruit blanc d'espérance nulle et de variance 1.\newline

\question{1} Le processus est-il asymptotiquement stationnaire au second ordre
et inversible~? Justifiez votre réponse.\newline

On suppose  que les conditions initiales sont telles que le
processus est stationnaire au second ordre.\newline

\question{2} Quelles sont les implications de cette hypothèse sur les
moments d'ordre 1 et 2~? \question{3} Calculez l'espérance (on notera
$\mu$ l'espérance). \question{4} Calculez les autocovariances d'ordre
0 et 1 (on notera $\gamma(0)$ et $\gamma(1)$). \question{5} Calculez
l'autocovariance d'ordre 2 (on notera $\gamma(2)$). \question{6}
Calculez l'autocovariance d'ordre $h$ (on notera $\gamma(h)$) pour tout
$h>2$.\newline

\bigskip

\exercice{4} Soit le processus AR(2) :
\[
Y_t = c + (\rho_1+\rho_2)Y_{t-1} - \rho_1\rho_2 Y_{t-2} + \varepsilon_t
\]
avec $(\varepsilon_t, t\in\mathbb Z)$ un bruit blanc d'espérance nulle et de
variance $\sigma^2$, $|\rho_1|<1$, $|\rho_2|<1$ et $\rho_1\neq\rho_2$.
\question{1} Montrer que le processus est asymptotiquement stationnaire au
second ordre. \question{2} Écrire la vraisemblance exacte du processus (en
utilisant le théorème de Bayes et détaillant le plus possible l'expression de la
vraisemblance). \question{3} Écrire la vraisemblance conditionnelle du
processus, en prenant soin d'expliquer son intérêt par rapport à la
vraisemblance exacte.\newline

\bigskip

\exercice{5} Soit le processus stochastique défini par~:
\[
  \begin{cases}
    y_t &= \varphi y_{t-1} + u_t\\
    u_t &= \varepsilon_t - \theta \varepsilon_{t-1}
  \end{cases}
\]
où $\varepsilon_t$ est un bruit blanc d'espérance nulle et de variance
$\sigma_{\varepsilon}2$, avec $\varphi$ et $\theta$ plus petits que 1 en valeur
absolue. \question{1} Montrer qu'il est possible d'écrire ce processus
stochastique sous la forme $y_t = \sum_{i=0}^{\infty}\psi_i\varepsilon_{t-i}$,
en explicitant $\psi_i$. \question{2} Calculer la variance de $y_t$ à partir de
la représentation alternative obtenue à la question 1.

\end{document}

%%% Local Variables:
%%% mode: latex
%%% TeX-master: t
%%% End:
