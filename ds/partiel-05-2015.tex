% stephane [DOT] adjemian [AT] univ [DASH] lemans [DOT] fr
\documentclass[10pt,a4paper,notitlepage,onecolumn]{article}
\usepackage{amsmath}
\usepackage{amssymb}
\usepackage{amsbsy}
\usepackage[T1]{fontenc}
\usepackage[utf8x]{inputenc}
\usepackage{palatino}
\usepackage{scrtime}
\usepackage[frenchb]{babel}
\usepackage{float}

%%%%%%%%%%%%%%%%%%%%%%%%%%%%%%%%%%%%%%%%%%%%%%%%%%%%%%%%%%%%%%%%%%%%%%%%%%%%%%%%%%%%%%%%%%%%%%%%%%%%

\newcommand{\exercice}[1]{\textsc{\textbf{Exercice}} #1}
\newcommand{\question}[1]{\textbf{(#1)}}
\setlength{\parindent}{0cm}



\begin{document}


\title{\textsc{Séries temporelles}}
\author{\textsc{Université du Maine (Partiel, L3)}}
\date{}


\maketitle


\exercice{1} Donnez l'expression de la vraisemblance exacte d'un
processus MA(1) d'espérance nulle. On notera $\mathcal Y_T \equiv
\{y_1,y_2,\dots,y_T\}$ l'échantillon et on supposera que les
innovations sont normalement distribuées d'espérance nulle et de
variance $\sigma_{\epsilon}^2$. Serait-il possible d'estimer les
paramètres de ce modèle par les MCO ?

\bigskip
\bigskip

\exercice{2} Supposons que $\{y_t,t\in\mathbb Z\}$ soit un ARMA($1,1$) de la forme :

\[
y_t = 1 + \frac{1}{2}y_{t-1} + \varepsilon_t - \frac{1}{4} \varepsilon_{t-1}
\]
avec $\varepsilon_t$ un bruit blanc d'espérance nulle et de variance 1.\newline

\question{1}   Le    processus   est-il   asymptotiquement
stationnaire  au   second  ordre   et  inversible~?   Justifiez  votre
réponse.\newline

On suppose maintenant que le processus est stationnaire au second
ordre.\newline

\question{2} Quelles sont les implications de cette hypothèse sur les
moments d'ordre 1 et 2 ? \question{3} Calculez l'espérance (on notera
$\mu$ l'espérance). \question{4} Calculez les autocovariances d'ordre
0 et 1 (on notera $\gamma(0)$ et $\gamma(1)$). \question{5} Calculez
l'autocovariance d'ordre 2 (on notera $\gamma(2)$). \question{6}
Calculez l'autocovariance d'ordre h (on notera $\gamma(h)$) pour tout
$h>2$.  \question{7} Définissez la fonction d'autocorrélation.

\bigskip
\bigskip

\exercice{3} Calculez les moments d'ordre 1 et 2 d'un processus ARMA(1,2) que nous supposerons stationnaire :
\[
y_t = c + \varphi y_{t-1} + \varepsilon_t - \theta_1 \varepsilon_{t-1} - \theta_2 \varepsilon_{t-2}
\]
où $\varepsilon_t$ est un bruit blanc d'espérance nulle et de variance $\sigma^2$.

\end{document}
