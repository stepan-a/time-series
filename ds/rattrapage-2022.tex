% stephane [DOT] adjemian [AT] univ [DASH] lemans [DOT] fr
\documentclass[10pt,a4paper,notitlepage,twocolumn]{article}
\usepackage{amsmath}
\usepackage{amssymb}
\usepackage{amsbsy}
\usepackage[T1]{fontenc}
\usepackage[utf8x]{inputenc}
\usepackage{palatino}
\usepackage{scrtime}
\usepackage[frenchb]{babel}
\usepackage{float}
\usepackage{nicefrac}

%%%%%%%%%%%%%%%%%%%%%%%%%%%%%%%%%%%%%%%%%%%%%%%%%%%%%%%%%%%%%%%%%%%%%%%%%%%%%%%%%%%%%%%%%%%%%%%%%%%%

\newcommand{\exercice}[1]{\textsc{\textbf{Exercice}} #1}
\newcommand{\question}[1]{\textbf{(#1)}}
\setlength{\parindent}{0cm}



\begin{document}


\title{\textsc{Séries temporelles}}
\date{Le \today\ à \thistime}


\maketitle

\exercice{1} Soient les fonctions d'autocovariance et de d'autocorrélation
partielle suivantes :
\begin{table}[H]
  \centering
  \begin{tabular}{r|cccccc}
    \hline\hline
    $h$ & 0 & 1 & 2 & 3 & 4 & 5 \\\hline
    $\gamma(h)$ & $\nicefrac{4}{3}$ & $\nicefrac{2}{3}$ & $\nicefrac{1}{3}$ & $\nicefrac{1}{6}$ & $\nicefrac{1}{12}$ & $\nicefrac{1}{24}$\\
    $r (h)$ & -- & 0,50 & 0,00 & 0,00 & 0,00 & 0,00\\
    \hline\hline
  \end{tabular}
\end{table}
\noindent Ces  fonctions sont générées par  un processus ARMA($p$,$q$)
avec $p\geq 0$, $q\geq 0$ et $p+q \leq 2$. En
justifiant  votre réponse,  déterminez la forme  du  processus qui  a
généré $\gamma(h)$ et $r(h)$. Que pouvez vous dire des paramètres de ce
modèle~?

\bigskip
\bigskip

\exercice{2} Soît $(Y_t,t\in\mathbb Z)$ un processus MA(1) inversible de moyenne
non nulle. On observe une réalisation de ce processus, un échantillon, que nous
noterons $\mathcal Y_T = \{y_1,y_2,\dots,y_T\}$. \question{1} Est-il possible
d'estimer ce modèle par les MCO ? Pourquoi ? \question{2} Écrire la
vraisemblance conditionnelle. \question{3} Discuter les conséquences de
l'hypothèse sur la condition initiale. Sous quelle hypothèse le choix de la
condition initiale n'a pas d'importance asymptotiquement ?


\bigskip
\bigskip

\exercice{3} Supposons que $\{y_t,t\in\mathbb Z\}$ soit un ARMA($1,1$) de la forme :
\[
y_t = \frac{2}{3}y_{t-1} + \varepsilon_t - \frac{1}{2} \varepsilon_{t-1}
\]
avec $\varepsilon_t$ un bruit blanc d'espérance nulle et de variance 1.\newline

\question{1} Le processus est-il asymptotiquement stationnaire au second ordre
et inversible~? Justifiez votre réponse.\newline

On suppose  que les conditions initiales sont telles que le
processus est stationnaire au second ordre.\newline

\question{2} Quelles sont les implications de cette hypothèse sur les
moments d'ordre 1 et 2 ? \question{3} Calculez l'espérance (on notera
$\mu$ l'espérance). \question{4} Calculez les autocovariances d'ordre
0 et 1 (on notera $\gamma(0)$ et $\gamma(1)$). \question{5} Calculez
l'autocovariance d'ordre 2 (on notera $\gamma(2)$). \question{6}
Calculez l'autocovariance d'ordre $h$ (on notera $\gamma(h)$) pour tout
$h>2$.

\bigskip
\bigskip

\exercice{4} Un modèle macroéconomique nous dit que la production est
déterminée par une fonction de production Cobb-Douglas ; le logarithme
de la production est une fonction linéaire du stock de
capital en logarithme:
\[
y_t = \alpha k_{t} + a_t
\]
où $a_t$ est le logarithme de la productivité, dont la dynamique est
donnée par un modèle AR(1):
\[
a_t = \rho a_{t-1} + \varepsilon_t
\]
avec $\varepsilon_t$ une variable aléatoire d'espérance nulle et de
variance $\sigma_{\varepsilon}^2$. Le paramètre $\rho$ est tel que la
productivité (en logarithme) est stationnaire, le paramètre
$\alpha\in]0,1[$ est l'élasticité de la production par rapport au
stock de capital physique. La dynamique de ce dernier résulte des
choix d'épargne des ménages. On admet qu'elle est donnée par:
\[
k_t = \eta_{kk} k_{t-1} + \eta_{ka}a_{t-1}
\]
\question{1} Montrer que $y_t$ est un processus ARMA(2,1).
\emph{\textbf{Indice :} Réécrire le modèle à l'aide de polynômes retard.}
\question{2} Calculer les racines de la partie autorégressive de ce processus
ARMA(2,1) et déduire une condition sur $\eta_{kk}$ assurant la stationarité de
la production.

\end{document}

%%% Local Variables:
%%% mode: latex
%%% TeX-master: t
%%% End:
