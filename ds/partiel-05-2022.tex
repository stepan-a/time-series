% stephane [DOT] adjemian [AT] univ [DASH] lemans [DOT] fr
\documentclass[10pt,a4paper,notitlepage,twocolumn]{article}
\usepackage{amsmath}
\usepackage{amssymb}
\usepackage{amsbsy}
\usepackage[T1]{fontenc}
\usepackage[utf8x]{inputenc}
\usepackage{palatino}
\usepackage{scrtime}
\usepackage[frenchb]{babel}
\usepackage{float}
\usepackage{nicefrac}

%%%%%%%%%%%%%%%%%%%%%%%%%%%%%%%%%%%%%%%%%%%%%%%%%%%%%%%%%%%%%%%%%%%%%%%%%%%%%%%%%%%%%%%%%%%%%%%%%%%%

\newcommand{\exercice}[1]{\textsc{\textbf{Exercice}} #1}
\newcommand{\question}[1]{\textbf{(#1)}}
\setlength{\parindent}{0cm}



\begin{document}


\title{\textsc{Séries temporelles}}
\author{\textsc{Université du Mans (Examen, L3)}}
\date{}


\maketitle

\exercice{1} Soient les fonctions d'autocovariance et de d'autocorrélation
partielle suivantes :
\begin{table}[H]
  \centering
  \begin{tabular}{r|cccccc}
    \hline\hline
    $h$ & 0 & 1 & 2 & 3 & 4 & 5 \\\hline
    $\rho (h)$ & $\nicefrac{4}{3}$ & $\nicefrac{2}{3}$ & $\nicefrac{1}{3}$ & $\nicefrac{1}{6}$ & $\nicefrac{1}{12}$ & $\nicefrac{1}{24}$\\
    $r (h)$ & -- & 0,50 & 0,00 & 0,00 & 0,00 & 0,00\\
    \hline\hline
  \end{tabular}
\end{table}
\noindent Ces  fonctions sont générées par  un processus ARMA($p$,$q$)
avec $p\geq 0$, $q\geq 0$ et $p+q \leq 2$. En
justifiant  votre réponse,  déterminez la forme  du  processus qui  a
généré $\rho(h)$ et $r(h)$. Que pouvez vous dire des paramètres de ce
modèle~?

\bigskip
\bigskip

\exercice{2} Soît $(Y_t,t\in\mathbb Z)$ un processus AR(1) stationnaire de
moyenne non nulle. On observe une réalisation de ce processus, un échantillon,
que nous noterons $\mathcal Y_T = \{y_1,y_2,\dots,y_T\}$. \question{1} Écrire la
vraisemblance exacte. \question{2} Écrire la vraisemblance conditionnelle.
\question{3} Montrez l'équivalence entre l'estimateur du maximum de
vraisemblance conditionnelle et l'estimateur des MCO.


\bigskip
\bigskip

\exercice{3} Supposons que $\{y_t,t\in\mathbb Z\}$ soit un ARMA($1,1$) de la forme :
\[
y_t = \frac{2}{3}y_{t-1} + \varepsilon_t - \frac{1}{2} \varepsilon_{t-1}
\]
avec $\varepsilon_t$ un bruit blanc d'espérance nulle et de variance 1.\newline

\question{1} Le processus est-il asymptotiquement stationnaire au second ordre
et inversible~? Justifiez votre réponse.\newline

On suppose  que les conditions initiales sont telles que le
processus est stationnaire au second ordre.\newline

\question{2} Quelles sont les implications de cette hypothèse sur les
moments d'ordre 1 et 2 ? \question{3} Calculez l'espérance (on notera
$\mu$ l'espérance). \question{4} Calculez les autocovariances d'ordre
0 et 1 (on notera $\gamma(0)$ et $\gamma(1)$). \question{5} Calculez
l'autocovariance d'ordre 2 (on notera $\gamma(2)$). \question{6}
Calculez l'autocovariance d'ordre $h$ (on notera $\gamma(h)$) pour tout
$h>2$.

\bigskip
\bigskip

\exercice{4} Soit le processus AR(2) :
\[
Y_t = c + (\rho_1+\rho_2)Y_{t-1} - \rho_1\rho_2 Y_{t-2} + \varepsilon_t
\]
avec $(\varepsilon_t, t\in\mathbb Z)$ un bruit blanc d'espérance nulle et de
variance $\sigma^2$, $|\rho_1|<1$, $|\rho_2|<1$ et $\rho_1\neq\rho_2$.
\question{1} Montrez que le processus est asymptotiquement stationnaire au
second ordre. On supposera dans la suite que le processus est stationnaire au
second ordre. \question{2} Calculez l'espérance inconditionnelle de $Y_t$.
\question{3} Calculez la fonction les autocovariances d'ordre 0, 1 et 2.
\question{4} Donnez une expression récursive de l'autocovariance d'ordre $h$
(c'est-à-dire exprimez $\gamma(h)$ en fonction de $\gamma(h-1)$ et
$\gamma(h-2)$). \question{5} Calculez le terme général de la récurrence d'ordre
deux caractérisant la fonction d'autocovariance et concluez sur le comportement
asymptotique de la fonction d'autocovariance (c'est-à-dire quand $h$ tend vers
l'infini).


\end{document}

%%% Local Variables:
%%% mode: latex
%%% TeX-master: t
%%% End:
