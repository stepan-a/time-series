% stephane [DOT] adjemian [AT] univ [DASH] lemans [DOT] fr
\documentclass[10pt,a4paper,notitlepage,twocolumn]{article}
\usepackage{amsmath}
\usepackage{amssymb}
\usepackage{amsbsy}
\usepackage[T1]{fontenc}
\usepackage[utf8x]{inputenc}
\usepackage{palatino}
\usepackage{scrtime}
\usepackage[frenchb]{babel}
\usepackage{float}

%%%%%%%%%%%%%%%%%%%%%%%%%%%%%%%%%%%%%%%%%%%%%%%%%%%%%%%%%%%%%%%%%%%%%%%%%%%%%%%%%%%%%%%%%%%%%%%%%%%%

\newcommand{\exercice}[1]{\textsc{\textbf{Exercice}} #1}
\newcommand{\question}[1]{\textbf{(#1)}}
\setlength{\parindent}{0cm}

\begin{document}

\title{\textsc{Séries temporelles}}
\author{\textsc{Université du Maine (Rattrapage, L3)}}
\date{}

\maketitle

\exercice{1} Soit le processus stochastique défini par :
\[
  y_t = \varepsilon_t - \theta \varepsilon_{t-2}
\]
avec $\varepsilon_t$ un bruit blanc d'espérance nulle et de variance
$\sigma_{\varepsilon}^2$. \textbf{(1)} Quel est le nom de ce
processus ? \textbf{(2)} Est-il possible d'estimer les paramètres de
ce modèle ($\theta$ et $\sigma_{\varepsilon}^2$) par les MCO ?
\textbf{(3)} On note $\mathcal Y_T = \{y_1,\dots,y_T\}$ l'échantillon,
donner l'expression de la vraisemblance exacte. \textbf{(4)} Donner
l'expression de la vraisemblance conditionnelle.

\bigskip
\bigskip

\exercice{2} Supposons que $\{y_t,t\in\mathbb Z\}$ soit un ARMA($1,1$) de la forme :

\[
y_t = 1 + \frac{2}{3}y_{t-1} + \varepsilon_t - \frac{1}{3} \varepsilon_{t-1}
\]
avec $\varepsilon_t$ un bruit blanc d'espérance nulle et de variance 1.\newline

\question{1}   Le    processus   est-il   asymptotiquement
stationnaire  au   second  ordre   et  inversible~?   Justifier  votre
réponse.\newline

On suppose maintenant que le processus est stationnaire au second
ordre.\newline

\question{2} Quelles sont les implications de cette hypothèse sur les
moments d'ordre 1 et 2 ? \question{3} Calculer l'espérance (on notera
$\mu$). \question{4} Calculer les autocovariances d'ordre
0 et 1 (on notera $\gamma(0)$ et $\gamma(1)$). \question{5} Calculer
l'autocovariance d'ordre 2 (on notera $\gamma(2)$). \question{6}
Calculer l'autocovariance d'ordre h (on notera $\gamma(h)$) pour tout
$h>2$.  \question{7} Définisser la fonction d'autocorrélation.

\bigskip
\bigskip

\exercice{3} Soit le processus ARMA(2,1) :
\[
y_t = c + \varphi_1 y_{t-1} + \varphi_2 y_{t-2} + \varepsilon_t - \theta \varepsilon_{t-1}
\]
où $\varepsilon_t$ est un bruit blanc d'espérance nulle et de variance
$\sigma^2$ et $c$, $\varphi_1$, $\varphi_2$ et $\theta$ sont des
paramètres réels. \textbf{(1)} Donner les conditions sur $c$,
$\varphi_1$, $\varphi_2$ et $\theta$ qui assurent la stationnarité du
processus stochastique. \textbf{(2)} Donner les conditions sur $c$,
$\varphi_1$, $\varphi_2$ et $\theta$ sous lesquelles le processus
stochastique est inversible. \textbf{(3)} Sous quelle condition ce
processus ARMA(2,1) est bien une représentation minimale du processus
stochastique ? \textbf{(4)} Sous l'hypothèse de stationnarité,
calculer l'espérance du processus stochastique. \textbf{(5)} Sous la
même hypothèse, calculer la fonction d'autocovariance.

\end{document}
