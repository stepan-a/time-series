\documentclass[12pt,a4paper,notitlepage]{article}
\synctex=1
\usepackage{amsmath}
\usepackage{amssymb}
\usepackage{amsbsy}
\usepackage[T1]{fontenc}
\usepackage[utf8x]{inputenc}
\usepackage{palatino}
\usepackage{scrtime}
\usepackage[frenchb]{babel}
\usepackage{float}
\usepackage{nicefrac}
\newcommand{\exercice}[1]{\textsc{\textbf{Exercice}} #1}
\newcommand{\question}[1]{\textbf{(#1)}}
\setlength{\parindent}{0cm}



\begin{document}


\title{\textsc{Séries temporelles}\\(Éléments de correction)}
\author{\textsc{Université du Maine (Partiel, L3)}}
\date{}


\maketitle

\exercice{1} Dans le cas du premier processus stochastique, $y_{t}$ et
$y_{t-2}$ ne sont pas indépendants car $y_{t-1}$ dépend de $y_{t-2}$
(et bien sûr $\varepsilon_{t-1}$) via $y_{t-1}$. Il suffit d'écrire la
définition du processus en $t-1$ pour s'en convaincre. Dans le cas du
deuxième processus, il y a indépendance entre $y_{t}$ et
$y_{t-2}$. Dans ce cas, puisque le processus stochastique est un
MA(1), $y_{t}$ dépend de $\varepsilon_{t}$ et $\varepsilon_{t-1}$
alors que $y_{t-2}$ dépend de $\varepsilon_{t-2}$ et
$\varepsilon_{t-3}$. Il n'y a donc pas d'innovations communes ; comme
ces innovations sont indépendantes (${\varepsilon_{t}}$ est un bruit
blanc) on a nécessairement l'indépendance de $y_{t}$ et
$y_{t-2}$. Enfin, dans le cas du troisième processus stochastique, on
n'a pas la propriété d'ndépendance. En notant qu'il s'agit d'un MA(2),
on voit que $y_{t}$ et $y_{t-2}$ dépendent de $\varepsilon_{t-2}$.\newline

\bigskip

\exercice{2} Reportez-vous au cours... Vous devez trouver que
l'esimateur des MCO est identique à l'estimateur du maximum de
vraisemblance conditionnelle, et noter que celui converge vers
l'estimateur du maximum de vraisemblance exacte lorque la taille de
l'échantillon tend vers l'infini.\newline

\bigskip

\exercice{3} \textbf{(1)} Le processus est asymptotiquement
stationnaire au second ordre car les racines du polynôme retard
auto-régressif ($
\nicefrac{1}{2}\pm\frac{\sqrt{11}}{2}$) sont supérieures à 1 en module. Le processus est
inversible car la racine du polynôme retard moyenne mobile (3) est
supérieure à 1 en module. On note que les racines sont différentes,
nous travaillons donc bien sur la représentation minimale d'une
ARMA(2,1). \textbf{(2)} Les moments d'ordres 1 et 2 sont donc
invariants. \textbf{(3)} L'espérance est donnée par :
\[
\mathbb E [y_t] = \frac{1}{1-\frac{1}{3}+\frac{1}{3}} = 1 
\]
\textbf{(4)} On doit trouver :
\[
\gamma(0) = \frac{17}{15} 
\]
\[
\gamma(1) = \frac{1}{30}
\]
et
\[
\gamma(2) = -\frac{11}{30}
\]
\textbf{(5)} L'autocovariance d'ordre 3 est :
\[
\gamma(2) = -\frac{2}{15}
\]
\textbf{(6)} Plus généralement, on a :
\[
\gamma(h) = \frac{1}{3}\gamma(h-1)-\frac{1}{3}\gamma(h-2)
\]
dès lors que l'on a passé l'influcence de la partie MA (ie $h>1$).
\textbf{(7)} La fonction d'auto-corrélation est une normalisation de la fonction d'auto-covariance, de façon générale on a :
\[
\rho(h) = \frac{\gamma(h)}{\gamma(0)}
\]

\bigskip

\exercice{4} Une solution détaillée est disponible sur ma page (dans la section vieilleries en vrac).\newline

\bigskip

\exercice{5} \textbf{(1)} On peut réécrire la loi d'évolution du stock de capital physique en utilisant des polynômes retard. On a :
\[
  (1-\eta_{kk}L)k_{t} = \eta_{ka} a_{t-1}
\]
Par ailleurs on a :
\[
(1-\rho L) a_{t} = \varepsilon_{t} 
\]
Soit par inversion du polynôme retard et substitution dans la loi d'évolution du stock de capital physique :
\[
  (1-\eta_{kk}L)k_{t} = \eta_{ka} (1-\rho L)^{-1} \varepsilon_{t-1}
\]
soit encore en multipliant les deux membres par $(1-\rho L)$  :
\[
(1-\eta_{kk}L)(1-\rho L)k_{t} = \eta_{ka} \varepsilon_{t-1}
\]
Le stock de capital est donc un processus stochastipe de type AR(2). En substiuant dans la fonction de production (en log) il vient :
\[
y_{t} = \alpha (1-\eta_{kk}L)^{-1}(1-\rho L)^{-1} \eta_{ka} \varepsilon_{t-1} + (1-\rho L)^{-1}\varepsilon_{t}
\]
Soit en inversant le polynôme retard d'ordre 2 :
\[
(1-\eta_{kk}L)(1-\rho L) y_{t} = \alpha \eta_{ka} \varepsilon_{t-1} + (1-\eta_{kk})\varepsilon_{t}
\]
ou encore :
\[
(1-\eta_{kk}L)(1-\rho L) y_{t} = \varepsilon_{t} - (\eta_{kk}-\alpha\eta_{ka}) \varepsilon_{t-1}
\]
et en développant le polynôme retard d'ordre deux sur le membre de gauche :
\[
y_{t} = (\eta_{kk}+\rho)y_{t-1} - \eta_{kk}\rho y_{t-2} + \varepsilon_{t} - (\eta_{kk}-\alpha\eta_{ka}) \varepsilon_{t-1}
\]
Il s'agit bien dn processus ARMA(2,1). \textbf{(2)} De ce qui précède, on voit directement que les racines du polynôme retard associé à la partie auto-régressive sont $\eta_{kk}$ et $\rho$. La condition de stationarité est donc $\eta_{kk}$ plus peit que 1 (en valeur absolue), c'est-à-dire la stabilité de la dynmique du capital physique (en plus de celle de la productivité). 




\end{document}

%%% Local Variables:
%%% mode: latex
%%% TeX-master: t
%%% End:
