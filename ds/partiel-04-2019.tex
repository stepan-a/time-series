% stephane [DOT] adjemian [AT] univ [DASH] lemans [DOT] fr
\documentclass[10pt,a4paper,notitlepage,twocolumn]{article}
\usepackage{amsmath}
\usepackage{amssymb}
\usepackage{amsbsy}
\usepackage[T1]{fontenc}
\usepackage[utf8x]{inputenc}
\usepackage{palatino}
\usepackage{scrtime}
\usepackage[frenchb]{babel}
\usepackage{float}
\usepackage{nicefrac}

%%%%%%%%%%%%%%%%%%%%%%%%%%%%%%%%%%%%%%%%%%%%%%%%%%%%%%%%%%%%%%%%%%%%%%%%%%%%%%%%%%%%%%%%%%%%%%%%%%%%

\newcommand{\exercice}[1]{\textsc{\textbf{Exercice}} #1}
\newcommand{\question}[1]{\textbf{(#1)}}
\setlength{\parindent}{0cm}



\begin{document}


\title{\textsc{Séries temporelles}}
\author{\textsc{Université du Maine (Examen, L3)}}
\date{}


\maketitle

\exercice{1} Soient les fonctions d'auto-corrélation et de corrélation
partielle suivantes :
\begin{table}[H]
  \centering
  \begin{tabular}{r|cccccc}
    \hline\hline
    $h$ & 0 & 1 & 2 & 3 & 4 & 5 \\\hline
    $\rho (h)$ & 1 & 0,70 & 0,49 & 0,34 & 0,24 & 0,17\\
    $r (h)$ & -- & 0,70 & 0,00 & 0,00 & 0,00 & 0,00\\
    \hline\hline
  \end{tabular}
\end{table}
\noindent Ces  fonctions sont générées par  un processus ARMA($p$,$q$)
avec $p$ égal à zéro ou un, $q$ égal  à zéro ou un et $p+q \leq 2$. En
justifiant  votre réponse,  déterminer  la forme  du  processus qui  a
généré $\rho(h)$ et $r(h)$. Que  pouvez-vous dire des paramètres de ce
modèle~?

\bigskip
\bigskip

\exercice{2}  Montrer,  pour  un  modèle  AR(1),  l'équivalence  entre
l'estimateur  des  MCO et  l'estimateur  du  maximum de  vraisemblance
conditionnelle. Donner l'expression de l'estimateur des MCO de $\rho$,
le paramètre autorégressif.

\bigskip
\bigskip

\exercice{3} Supposons que $\{y_t,t\in\mathbb Z\}$ soit un ARMA($1,1$) de la forme :

\[
y_t = \frac{1}{2}y_{t-1} + \varepsilon_t - \frac{1}{3} \varepsilon_{t-1}
\]
avec $\varepsilon_t$ un bruit blanc d'espérance nulle et de variance 1.\newline

\question{1}   Le    processus   est-il   asymptotiquement
stationnaire  au   second  ordre   et  inversible~?   Justifier  votre
réponse.\newline

On suppose  que les conditions initiales sont  telles que le
processus est stationnaire au second ordre.\newline

\question{2} Quelles sont les implications de cette hypothèse sur les
moments d'ordre 1 et 2 ? \question{3} Calculer l'espérance (on notera
$\mu$ l'espérance). \question{4} Calculer les autocovariances d'ordre
0 et 1 (on notera $\gamma(0)$ et $\gamma(1)$). \question{5} Calculer
l'autocovariance d'ordre 2 (on notera $\gamma(2)$). \question{6}
Calculer l'autocovariance d'ordre $h$ (on notera $\gamma(h)$) pour tout
$h>2$.

\bigskip
\bigskip

\exercice{4} Soit le modèle état mesure suivant :
\[
  \begin{split}
    y_t &= z_t + \varepsilon_t\\
    z_t &= z_{t-1} + \eta_t
  \end{split}
\]
avec $\varepsilon_t \equiv BB(0,\sigma_{\varepsilon}^2)$ et
$\eta_t \equiv BB(0,\sigma_{\eta}^2)$. Le second processus,
$\left\{z_t\right\}_{t\in \mathbb Z}$ est une marche aléatoire,
$\left\{y_t\right\}_{t\in \mathbb Z}$ et une marche aléatoire
<<bruitée>> et on notera
$\kappa = \nicefrac{\sigma_{\eta}^2}{\sigma_{\varepsilon}^2}$ le ratio
signal-bruit. \question{1} Montrer que la variance de
$\left\{y_t\right\}_{t\in \mathbb Z}$ n'est pas définie. Donner
l'intuition. \question{2} Calculer la fonction d'autocovariance de
$\left\{\Delta y_t\right\}_{t\in \mathbb Z}$. \question{3} Conclure
sur la nature du processus
$\left\{\Delta y_t\right\}_{t\in \mathbb Z}$. \question{4} Identifier
les paramètres du modèle.



\end{document}

%%% Local Variables:
%%% mode: latex
%%% TeX-master: t
%%% End:
