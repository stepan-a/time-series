% stephane [DOT] adjemian [AT] univ [DASH] lemans [DOT] fr
\documentclass[10pt,a4paper,notitlepage,onecolumn]{article}
\usepackage{amsmath}
\usepackage{amssymb}
\usepackage{amsbsy}
\usepackage[T1]{fontenc}
\usepackage[utf8x]{inputenc}
\usepackage{palatino}
\usepackage{scrtime}
\usepackage[frenchb]{babel}
\usepackage{float}

%%%%%%%%%%%%%%%%%%%%%%%%%%%%%%%%%%%%%%%%%%%%%%%%%%%%%%%%%%%%%%%%%%%%%%%%%%%%%%%%%%%%%%%%%%%%%%%%%%%%

\newcommand{\exercice}[1]{\textsc{\textbf{Exercice}} #1}
\newcommand{\question}[1]{\textbf{(#1)}}
\setlength{\parindent}{0cm}



\begin{document}


\title{\textsc{Séries temporelles}}
\author{\textsc{Université du Maine (Examen, L3)}}
\date{}


\maketitle

\thispagestyle{empty}

\exercice{1} Soit un processus stochastique AR(1) avec constante,
écrire la vraisemblance conditionnelle et déterminer l'estimateur du
maximum de vraisemblance du paramètre autorégressif sur la base de
cette expression de la vraisemblance. L'estimateur est-il biaisé~?
Pourquoi~? Écrire la vraisemblance exacte. Est-il possible d'obtenir
une expression analytique de l'estimateur du maximum de vraisemblance
sur la base de cette expression de la vraisemblance~?



\bigskip\bigskip



\exercice{2} Soit le processus AR(2) défini par~:
\[
  Y_t = \frac{1}{2}Y_{t-1} - \frac{1}{4}Y_{t-2} + \varepsilon_t
\]
avec $\varepsilon_t$ une variable aléatoire normalement distribuée (d'espérance nulle et de variance 1) indépendante du passé de $Y$ vérifiant $\varepsilon_t \perp \varepsilon_s$ pour tout $t\neq s$.\newline

\question{1} Montrer que ce processus est asymptotiquement stationnaire.\newline

\question{2} Calculer les moments d'ordre 1 et 2 du processus.\newline

\question{3} Déterminer les conditions sous lesquelles le processus est stationnaire au second ordre.\newline



\bigskip\bigskip



\exercice{3} Calculez les moments d'ordre 1 et 2 d'un processus ARMA(1,2) que nous supposerons stationnaire~:
\[
y_t = c + \varphi y_{t-1} + \varepsilon_t - \theta_1 \varepsilon_{t-1} - \theta_2 \varepsilon_{t-2}
\]
où $\varepsilon_t$ est un bruit blanc d'espérance nulle et de variance $\sigma^2$.

\end{document}
